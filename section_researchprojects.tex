\sectionTitle{Key research projects}{\faKey}

\textbf{Mechanical Metamaterials, Non-Affine Deformations and Deployable Structures } \hfill{August 2018 - Present}\\
\textbf{Guide: Prof. Mandar M. Inamdar, Civil Engineering Department} \hfill{IIT Bombay} \\
\textbf{Description:}{ Project involved study of metamaterials, origami, topological mechanics, lattice mechanics, dislocations, non-affine deformations, form-finding, and deployable structures. Current work is focused on the design of active materials.}
\begin{itemize}
    \setlength\itemsep{0.7mm}
    \item Performed static and kinematic analysis of frameworks, followed by interpretation of fundamental subspaces of associated equilibrium and kinematic matrix to identify the states of self-stress and finite and infinitesimal mechanisms present. 
    \item Developed origami-based paper models of collapsible hollow cylindrical members with hexagonal unit cells and analyzed them at different stages of deployment, followed by classification based on the energy profile as they deployed.
    \item Worked on infinitely repetitive structures as present in crystal lattices and the associated reciprocal lattice, identified zero-energy modes, states of self stress, actuation characteristic, strain producing and non-strain producing deformations.
    \item Developing a Julia program to generate simple lattices and grids, studying their dynamic behavior and finding their forms under different stimulation such as change in natural length, states of self stress, and temperature variations.
\end{itemize}


\textbf{Stress Ratio based Structural Optimization using Genetic Algorithm} \hfill{July 2017 - August 2018}\\
\textbf{Guide: Prof. Yogesh M. Desai, Prof. Venkata S. K. Delhi, Department of Civil Engineering} \hfill{IIT Bombay} \\
\textbf{Description: }{Project involved optimizing a Fibre-Reinforced Plastic (FRP) based cooling tower with 13,000+ members using GA.}
\begin{itemize}
    \setlength\itemsep{0.7mm}
    \item Learned and implemented various unimodal and multimodal optimization algorithms including Simplex method, Newton-Ralphson, Conjugate Gradient, Sequential Quadratic Programming, Simulated Annealing and Genetic Algorithm.
    \item Utilized the concepts of symmetry and anti-symmetry to minimize the number of design variables and computational effort.
    \item Employed stress-based Member Utilization Ratio for evaluation of population fitness function and reproduced population.
    \item Used component object modelling library of STAAD.Pro (OpenSTAAD) for analyzing and iterating over the structure.
\end{itemize}

