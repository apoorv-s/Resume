\sectionTitle{Technical Projects}{\faWrench}

\textbf{Team Shunya} \hfill{July 2017 - Present}\\
\textbf{Chief Faculty Advisor: Prof. Rangan Banerjee, Department of Energy Science \& Engineering} \hfill{IIT Bombay} \\
{Team Shunya is a team of 9 faculty and 50+ students from across the disciplines representing India in Solar Decathlons, collegiate competition conducted by U.S. Dept of Energy in which teams design and build highly efficient buildings powered by renewable.}
\begin{itemize}
    \item {\textbf{Project LEAF, Solar Decathlon Europe - 2021} \hfill{August 2018 - Present}\\
        \textbf{Description:}{ Project LEAF is the current endeavor of the team aimed at providing sustainable housing solutions in tier II Indian cities, that are currently facing the waxing pressure of urbanization, by extending the existing buildings vertically.}\vspace{1mm}\\
        \textbf{Designation: Structure \&  Materials Subsystem Head}
        \begin{itemize}
            \setlength\itemsep{0.7mm}
            \item Working with a team of 10+ students to ideate and design structural framework and examining the applicability of Cold Formed Steel (CFS) and Fibre Reinforced Plastic (FRP) based construction practices to achieve sufficient lightness.
            \item Developed 3D printed representative deployable units that are currently being scaled for installation on the load-carrying frame and will enhance the efficiency and ease of construction.
            \item Classifying and selecting materials for auxiliary units of the house based on embodied energy and carbon footprint.
            \item Responsible for the integration of designs from architectural and technical subsystems of the team.
            \item Involved in ideation of the project's concept, structuring of the team, and establishment of team management.
        \end{itemize}}
        \vspace{7mm}
    \item {\textbf{Project Solarise, Solar Decathlon China - 2018} \hfill{July 2017 - August 2018}\\
        \textbf{Description:}{ Project Solarise was designed to solve the problem of haphazard housing widely evident in Indian cities. A prototype that can be stacked in horizontal as well as the vertical plane with planned community spaces was developed.}\vspace{1mm}\\
        \textbf{Designation: Design Engineer, Structural Subsystem}
        \begin{itemize}
            \setlength\itemsep{0.7mm}
            \item Represented India at Solar Decathlon China 2018 in Dezhou, China, by constructing an 1800 sq-ft solar-powered net-positive energy house within 12 days. The team received the Best Participation Award for its efficiency and commitment.
            \item Involved in Designing of Structure and selection of materials to ensure modularity and sustainability of the house.
            \item Documented phases involved in the design and construction process to be presented as competition deliverables.
        \end{itemize}}   
\end{itemize}{}


%\textbf{IITB Racing} \hfill{July 2017 - August 2018}\\
%\textbf{Chief Faculty Advisor: Prof. Amber Shrivastava, Department of Mechanical Engineering} \hfill{IIT Bombay} \\
%\textbf{Description:}{}\\
%\textbf{Designation:}
%\begin{itemize}
%    \item 
%\end{itemize}